\documentclass{llncs}

\usepackage{url}
\usepackage{graphicx}
\usepackage{trackchanges}

\begin{document}

\title{Designing and Developing Context-Aware Mobile Mashups: the CAMUS approach}
\author{Fabio Corvetta \and Maristella Matera \and Riccardo Medana \and Elisa Quintarelli \and Vincenzo Rizzo \and Letizia Tanca}

\authorrunning{Fabio Corvetta et al.} % abbreviated author list (for running head)

%%%% list of authors for the TOC (use if author list has to be modified)
\tocauthor{Valerio Cassani, Stefano Gianelli, Maristella Matera, Riccardo Medana, Elisa Quintarelli, Letizia Tanca and Vittorio Zaccaria}

\institute{Politecnico di Milano\\\textit{Dipartimento di Elettronica, Informazione e Bioingegneria}\\
\email{valerio.cassani@mail.polimi.it, stefano.gianelli@mail.polimi.it, maristella.matera@polimi.it, r.medana@gmail.com, elisa.quintarelli@polimi.it,  letizia.tanca@polimi.it, vittorio.zaccaria@polimi.it}\\
}

\maketitle              % typeset the title of the contribution

\begin{abstract}

Given the plethora of data and services today available online, it is
often difficult to identify on-the-fly the information or the
applications that are appropriate for a given context of use. This is
even more true in the mobile scenario, where device resources (memory,
computational power, transmission budget) are still limited and
therefore data filtering and tailoring are required features. Given
this evidence, our research focuses on methods and tools for the
design and development of Context-Aware Mobile mashUpS (CAMUS). CAMUS
apps dynamically collect and integrate data from documental, social
and Web resources filtered on the basis of the current context, and
further adapt the integrated content to the users' situational
needs. The underlying application paradigm therefore overcomes the
limits posed by pre-packaged apps and offers to the final users
flexible and personalized applications whose structure and content may
even emerge at runtime based on the actual user needs and situation of
use.
This paper presents a design method and an accompanying platform for
the development of CAMUS apps. The approach is characterized by the
role given to context as a first-class modeling dimension able to
support i) the identification of the most adequate resources that can
satisfy the users' situational needs and ii) the consequent tailoring
at runtime of the provided data and functions. Context-based
abstractions are exploited to generate models specifying how data
returned by selected services have to be integrated and visualized by
means of integrated views. Thanks to the adoption of Model-Driven
Engineering (MDE) techniques, these models finally drive the flexible
execution of the final mobile app on target mobile devices. A
prototype of the platform, making use of novel and advanced Web and
mobile technologies, is also illustrated.

\keywords{Mobile Mashup, Mashup Modeling, Context Modeling, Context-aware Mobile Applications, GraphQL}
\end{abstract}

\section{Introduction}
\input{Sections/Introduction}

\section{Rationale and Background}
\input{Sections/RationaleBackground}

\section{Platform Organization}
\input{Sections/Platform}

\section{Conclusions}
\input{Sections/Conclusions}

\section{Acknowledgments}



%
% ---- Bibliography ----
%
\bibliographystyle{splncs}
%\bibliography{biblio_MM}

\begin{thebibliography}{1}

\bibitem{DBLP:journals/vlc/ArditoCDLMPP14}
Ardito, C., Costabile, M.F., Desolda, G., Lanzilotti, R., Matera, M., Piccinno,
  A., Picozzi, M.:
\newblock User-driven visual composition of service-based interactive spaces.
\newblock J. Vis. Lang. Comput. \textbf{25}(4) (2014)  278--296

\bibitem{DBLP:journals/tlsdkcs/BianchiniCAFQT14}
Bianchini, D., Castano, S., De Antonellis, V., Ferrara, A., Quintarelli, E., Tanca, L.:
\newblock {RUBIK:} Proactive, Entity-Centric and Personalized Situational Web Application Design.
\newblock T. Large-Scale Data-and Knowledge-Cent. Syst., \textbf{13} (2014) 123--157

\bibitem{DBLP:journals/cacm/BolchiniCOQRST09}
Bolchini, C., Curino, C., Orsi, G., Quintarelli, E., Rossato, R., Schreiber,
  F.A., Tanca, L.:
\newblock And what can context do for data?
\newblock {CACM} \textbf{52}(11) (2009)  136--140

\bibitem{DBLP:journals/debu/BolchiniOQST11}
Bolchini, C., Orsi, G., Quintarelli, E., Schreiber, F.A., Tanca, L.:
\newblock Context modeling and context awareness: steps forward in the
  context-addict project.
\newblock {IEEE} Data Eng. Bull. \textbf{34}(2) (2011)  47--54

\bibitem{journals/TWEB2015/CappielloMP15}
Cappiello, C., Matera, M., Picozzi, M.:
\newblock A {UI}-centric approach for the {End-User Development} of
  multi-device mashups.
\newblock ACM Trans. on Web (To appear) (2015)

\bibitem{DBLP:books/sp/DanielM14}
Daniel, F., Matera, M.:
\newblock Mashups - Concepts, Models and Architectures.
\newblock Data-Centric Systems and Applications. Springer (2014)

\end{thebibliography}
\end{document}

%%% Local Variables:
%%% mode: latex
%%% TeX-master: t
%%% End:
